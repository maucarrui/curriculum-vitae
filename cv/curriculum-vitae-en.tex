\documentclass{curriculum-vitae}

\begin{document}
  % The following command is to avoid warnings.
  \hypersetup{pageanchor=false}

  \name{Mauricio Carrasco Ruiz}

  \vspace{1em}

  \contactPersonal%
      {\faMapMarker \ Mexico City, Mexico}
      {\faMobile \ XX XXXX XXXX}
      {\faEnvelopeO \ maucarrui@gmail.com}

  \vspace{1em}

  \contactSocial%
      {\faGithub
       \ \href{https://github.com/maucarrui}{github.com/maucarrui}}
      {\faLinkedinSquare
       \ \href{https://linkedin.com/in/maucarrui}{linkedin.com/in/maucarrui}}

  \section{Education}
    \datedsubsection%
        {BSc in Computer Science}
        {2017-2022}

        School of Sciences, National Autonomous University of Mexico (UNAM),
        Mexico City, Mexico.

        Graduated with a grade of (9.77/10), corresponding to a 4.0 GPA.

        \textcolor{Blue}{\tit{In Process of presenting the thesis defense.}}

    \datedsubsection%
      {Computer Technician}
      {2016-2017}

      Specialized Technical Studies, UNAM, Mexico City, Mexico.

      \tit{Graduated with honors.}

  \section{Experience}
    \experiencesubsection%
        {Teacher Assistant Level ``A'' and ``B''}
        {August 2021 - February 2023}
        {School of Sciences, UNAM}

        Taught two hours per week in the classroom as a complement to the three
        hours peer week corresponding to the professor of the subject, in
        addition to grading and designing assignments, excercises and exams,
        among others. Courses taught:

        \vspace{0.5em}

        \course%
            {Graphs \& Games}
            {Semesters 2021-1, 2021-2, 2022-1, 2022-2, 2023-1.}
            {Introductory course to the Graph and Game Theory, where the main
              subject of study are the math structures known as Graphs and its
              properties.}

        \course%
            {Linear Algebra I}
            {Semester 2023-1.}
            {Introductory course to the Linear Algebra, where the main subject
              of study are vector spaces and linear transformations.}

        \course%
            {Graph Theory}
            {Semester 2020-4.}
            {Intermediate course of Graph Theory, where the main subject of
              study are the concepts of connection, eulerian paths, matching,
              colouring, and planarity.}

    \experiencesubsection%
      {SNI Level III Reasearcher Assistant}
      {September 2018 - August 2019}
      {National Council of Science and Technology (CONACYT)}

      Collaboration with the Center for Genomic Sciences of the UNAM, where I
      was responsible for the development of a Java pipeline known as
      \textsc{Graphene2Gephi}, which is able to extract a knowledge-graph given
      the context provided by \textsc{Graphene}, a tool which uses Natural
      Language Processing techniques, and in this way visualize in a more
      user-friendly way the most informative content of the biomedical
      literature.

    \experiencesubsection%
      {Jr. PHP and JavaScript Back-End Developer}
      {March 2017 - October 2017}
      {National Preparatory School No. 6, ``Antonio Caso''}

      Development and maintenance of the CRUD web application responsible for
      the management of schedules and administration of subjects and schools.

    \experiencesubsection%
      {Jr. Laravel PHP Full-Stack Developer}
      {June 2016 - November 2016}
      {Social Service performed at the Faculty of Engineering, UNAM}

      Development of an administration system for the \tit{Palace of Mines
        International Book Fair} where I developed the views that where shown to
      the client as well as its proper working with the data base.

  \section{Publications}

    \subsection{\tbf{Papers}}

    \begin{enumerate}
      \item
        Arroyo Fernández Ignacio, \tbf{Carrasco Ruiz Mauricio}, and Arias
        Aguilar José Aguilar. \tit{On the Possibility of Rewarding Structure
          Learning Agents: Mutual Information on Linguistic Random Sets},
        Workshop on Sets \& Partitions, NeurIPS, October 2019.

      \item
        Arroyo Fernández Ignacio, Forest Dominic, Torres Moreno Juan Manuel,
        \tbf{Carrasco Ruiz Mauricio}, Legeleux Thomas and Joanette Karen,
        ``Cyber-bullying Detection Task: The EBSI-LIA-UNAM system (ELU) at
        COLING'18 TRAC-1'' at the \tit{27th International Conference of
          Computational Linguistics (COLING 2018)}, August 2018.
    \end{enumerate}

    \subsection{\tbf{Books}}

    \begin{enumerate}
      \item
        \tbf{Digital Heritage: Computational Methods and Interactive Mediums to
          Study and Divulge the Cultural Heritage},
        \tit{Jiménez Badillo Diego, Arroyo Fernández Ignacio, Méndez Cruz Carlos
          Francisco, \tbf{Carrasco Ruiz Mauricio}, et al.},
        Instituto Nacional de Antropología e Historia, Secretaría de Cultura,
        262 p.

        ISBN : 978-607-539-597-5
    \end{enumerate}

  \section{Talks}

    \subsection{\tbf{Conferences}}

    \begin{enumerate}
      \item
        Arroyo Fernández Ignacio, \tbf{Carrasco Ruiz Mauricio}, y Méndez Cruz
        Carlos Francisco,
        ``Natural Language Processsing in the conservation of the cultural
        heritage: a knowledge-graph of the Popol Vuh.'',
        \tit{Tercer Coloquio: Desarrollo Tecnológico al Servicio del Patrimonio
          Cultural}, Mexico City, Mexico, October 2018.
    \end{enumerate}

  \section{Research Stays}
    \begin{itemize}
      \item
        \tbf{Center for Genomic Sciences, UNAM}, 2018.

        Research stay to study Natural Language Processing (NLP) and its
        application on the search for relationships between proteins and genes,
        and its relation with diseases and mutation contained in scientific
        papers.

      \item
        \tbf{Applied Mathematics and Systems Research Institute, UNAM},
        2017.

        Short research stay to study Evolutionary Computation and its most
        significant algorithms; a techincal report and a scientific poster were
        written to share the results.

    \end{itemize}

  \section{Acknowledgements and Awards}

    \begin{itemize}
      \item
        First place in the Object-Oriented Programming module of the
        \tit{Concursos Interpreparatorianos}, 2017. (Annual competition held by
        UNAM where all the UNAM incorporated high-schools participate).

      \item
        Second place in the Informatics and Technology category of the
        \tit{Concurso de Informes Técnicos de Estancias Cortas}, 2017. (Annual
        Competition held by UNAM where students participate by writting a
        technical report about the activities done in a summer internship
        collaborating with a researcher).

      \item
        Second place in the robotics category of the \tit{25th Concurso
          Universitario: La Feria de Ciencias, La Tecnología y La Innovación},
        2017. (Fair Science held by UNAM).

      \item
        Participated in the \tit{XVII Muestra Científica de Estancias Cortas},
        2017. (Annual Event held by UNAM where students show the results they
        obtained in research stays).

      \item
        First generation of the \tit{Taller de Ciencias para Jóvenes en Mérida}
        held by the Mathematics Research Center, UNAM. (Winter internship where
        participants get to meet and work with multiple researchers).
    \end{itemize}

  \section{Academic Projects}

    \begin{itemize}
      \item \tbf{H2O al Cubo}, 2017.

        Design, construction and programming of an aquatic robot built using
        recicled materials to recolect trash found on lakes and lagoons, it's
        controlled using an Android App and an Arduino device.

      \item \tbf{PREPArados}, 2016.

        Developed a web application game to improve the performance of the
        National Preparatory School students, which was implemented via
        Sails.js.

    \end{itemize}

  \section{Skills}

    \skill%
      {Programming Languages}
      {Java, C/C++, Python, PHP, JavaScript, and Go.}

    \vspace{0.5em}

    \skill%
      {Operating Systems}
      {Linux (Ubuntu, Arch Linux), Windows.}

    \vspace{0.5em}

    \skill%
      {Miscellaneous}
      {Web development (HTML, CSS, SVG, JQuery, node.js), database
        administration (POSTGRES, MariaDB, SQLITE, MySQL), Git, Unix/Bash,
        frameworks (Django, Sails.js, Laravel PHP), and \LaTeX.}

  \section{Languages}
  \lan{Spanish}{Native Speaker}
  \lan{English}{C1}
\end{document}
