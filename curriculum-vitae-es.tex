\documentclass{curriculum-vitae}

\begin{document}
  % The following command is to avoid warnings.
  \hypersetup{pageanchor=false}

  \name{Mauricio Carrasco Ruiz}

  \vspace{1em}

  \contactPersonal%
      {\faMapMarker \ CDMX, México}
      {\faMobile \ XX XXXX XXXX}
      {\faEnvelopeO \ maucarrui@gmail.com}

  \vspace{1em}

  \contactSocial%
      {\faGithub
       \ \href{https://github.com/maucarrui}{github.com/maucarrui}}
      {\faLinkedinSquare
       \ \href{https://linkedin.com/in/maucarrui}{linkedin.com/in/maucarrui}}

  \section{Formación Académica}
    \datedsubsection%
        {Licenciatura en Ciencias de la Computación}
        {2017-2022}

        Facultad de Ciencias, Universidad Nacional Autónoma de México (UNAM),
        CDMX, México.

        Graduado con un promedio de 9.77.

        \textcolor{Blue}{\tit{En proceso de realizar el examen profesional.}}

    \datedsubsection%
      {Técnico en Computación}
      {2016-2017}
      
      Estudios Técnicos Especializados de la UNAM, CDMX, México.
      
      \tit{Mención Honorífica}

  \section{Experiencia Laboral}
    \experiencesubsection%
        {Ayudante de Profesor de Asignatura Nivel ``A'' y ``B''}
        {Agosto 2021 - Febrero 2023}
        {Facultad de Ciencias, UNAM}

        Impartición de dos horas semanales en el aula como complemento de las
        tres horas semanales del profesor titular de la materia, además de
        calificar y diseñar tareas, prácticas y exámenes, entre otros. Cursos
        impartidos:

        \vspace{0.5em}

        \course%
            {Gráficas y Juegos}
            {Semestres 2021-1, 2021-2, 2022-1, 2022-2, 2023-1.}
            {Curso introductorio a la Teor\'ia de Gráficas y la Teor\'ia de
              Juegos, donde se estudia principalmente a la estructura
              matem\'atica conocida como Gr\'afica y sus propiedades.}

        \course%
            {Álgebra Lineal I}
            {Semestre 2023-1.}
            {Curso introductorio al Álgebra Lineal, donde se estudia
              principalmente a los espacios vectoriales y las transformaciones
              lineales.}

        \course%
            {Teoría de Gráficas}
            {Intersemestral 2020-4}
            {Curso avanzado de Teor\'ia de Gr\'aficas donde se estudia a
              profundidad los conceptos de conexidad, recorridos eulerianos,
              apareamientos, coloración, y planaridad.}

    \experiencesubsection%
      {Asistente de Investigador SNI Nivel III}
      {Septiembre 2018 - Agosto 2019}
      {Consejo Nacional de Ciencia y Tecnología (CONACYT)}

      Colaboración con el Centro de Ciencias Genómicas de la UNAM donde fui
      responsable del desarrollo de un sistema en Java conocido como
      \textsc{Graphene2Gephi}, el cual es capaz de extraer una red semántica
      dado el contexto brindado por \textsc{Graphene}, una herramienta que
      utiliza técnicas del Procesamiento del Lenguaje Natural, y de esta manera
      visualizar de una forma más amigable el contenido más informativo de la
      literatura biomédica.

    \experiencesubsection%
      {Desarrollador Junior Back-End de PHP y JavaScript}
      {Marzo 2017 - Octubre 2017}
      {Escuela Nacional Preparatoria No. 6 ``Antonio Caso''}

      Estuve en el desarrollo y mantenimiento de la aplicación web CRUD
      encargada del manejo de horarios y administración de materias y colegios.

    \experiencesubsection%
      {Desarrollador Junior Full-Stack de Laravel PHP}
      {Junio 2016 - Noviembre 2016}
      {Servicio Social realizado en la Facultad de Ingenieria, UNAM}

      Estuve en el desarrollo de un sistema de administración para \textit{La
        Feria Internacional del Libro del Palacio de Mineria} con el marco de
      trabajo \textsc{Laravel} donde trabaja tanto con las vistas que se le
      mostraban al cliente así como su correcto funcionamiento con la base de
      datos.

  \section{Publicaciones}

    \subsection{\tbf{Artículos}}

    \begin{enumerate}
      \item
        Arroyo Fernández Ignacio, \tbf{Carrasco Ruiz Mauricio}, y Arias Aguilar
        José Aguilar. \tit{On the Possibility of Rewarding Structure Learning
          Agents: Mutual Information on Linguistic Random Sets}, Workshop on
        Sets \& Partitions, NeurIPS, Octubre 2019.

      \item
        Arroyo Fernández Ignacio, Forest Dominic, Torres Moreno Juan Manuel,
        \tbf{Carrasco Ruiz Mauricio}, Legeleux Thomas y Joanette Karen,
        ``Cyber-bullying Detection Task: The EBSI-LIA-UNAM system (ELU) at
        COLING'18 TRAC-1'' en el \tit{27th International Conference of
          Computational Linguistics (COLING 2018)}, Agosto 2018.
    \end{enumerate}

    \subsection{\tbf{Libros}}

    \begin{enumerate}
      \item
        \tbf{Patrimonio Digital: Métodos Computacionales y Medios Interactivos
          para Estudiar y Divulgar el Patrimonio Cultural},
        \tit{Jiménez Badillo Diego, Arroyo Fernández Ignacio, Méndez Cruz Carlos
          Francisco, \tbf{Carrasco Ruiz Mauricio}, et al.},
        Instituto Nacional de Antropología e Historia, Secretaría de Cultura,
        262 p.

        ISBN : 978-607-539-597-5
    \end{enumerate}

  \section{Pláticas}

    \subsection{\tbf{Conferencias}}

    \begin{enumerate}
      \item
        Arroyo Fernández Ignacio, \tbf{Carrasco Ruiz Mauricio}, y Méndez Cruz
        Carlos Francisco,
        ``Procesamiento del lenguaje natural en la conservación de la herencia
        cultural: una red semántica del Popol Vuh'',
        \tit{Tercer Coloquio: Desarrollo Tecnológico al Servicio del Patrimonio
          Cultural}, CDMX, México, Octubre 2018.
    \end{enumerate}

  \section{Estancias de Investigación}
    \begin{itemize}
      \item
        \tbf{Centro de Ciencias Genómicas de la UNAM (CCG)}, 2018.
        
        Estancia de investigación realizada sobre el \tit{Procesamiento del
          Lenguaje Natural (NLP)} para su aplicación en la búsqueda de
        relaciones entre proteínas y genes, y su relación con enfermedades y
        mutaciones dentro de artículos científicos.

      \item 
        \tbf{Instituto de Investigaciones Matemáticas Aplicadas y Sistemas
          (IIMAS)}, 2017.
        
        Estancia corta de investigación realizada sobre el \tit{Cómputo
          Evolutivo} y sus algoritmos más significativos, en donde se elaboró un
        informe técnico y un cartel científico.

    \end{itemize}

  \section{Premios y Reconocimientos}

    \begin{itemize}
      \item
        Primer lugar en la modalidad Programación Orientada a Objetos de los
        \tit{Concursos Interpreparatorianos} de la UNAM, 2016-2017.

      \item
        Participación en la \tit{XVII Muestra Científica de Estancias Cortas},
        2017.

      \item
        Segundo lugar en el área de robótica en el \tit{XXV Concurso
          Universitario: La Feria de las Ciencias, la Tecnología y la
          Innovación} de la UNAM, 2016-2017.

      \item
        Primera generación del Taller de Ciencias para \tit{Jóvenes en Mérida}
        del Centro de Investigación de Matemáticas, 2016.
    \end{itemize}

  \section{Proyectos Académicos}

    \begin{itemize}
      \item \tbf{H2O al Cubo}, 2017.
        
        Diseño, construcción y programación de un robot acuático conformado de
        materiales reciclados para la recolección de basura en lagos y lagunas,
        el cual es controlado por una aplicación Android y un dispositivo
        Arduino.
        
      \item \tbf{PREPArados}, 2016.

        Desarrollo de un juego en una aplicación web para mejorar el desempeño
        de los estudiantes de la Escuela Nacional Preparatoria el cual fue
        implementado utilizando el marco de trabajo Sails.js.
      
    \end{itemize}

  \section{Conocimientos y Habilidades}
  
    \skill%
      {Lenguajes de Programación}
      {Java, C/C++, Python, PHP, JavaScript, y Go.}

    \vspace{0.5em}

    \skill%
      {Sistemas Operativos}
      {Linux (Ubuntu, Arch Linux), Windows.}

    \vspace{0.5em}

    \skill% 
      {Misceláneos}
      {Herramientas para el desarrollo de aplicaciones y/o sitios web (HTML,
        CSS, SVG, JQuery, node.js), administración de bases de datos (POSTGRES,
        MariaDB, SQLITE, MySQL), Git, Unix/Bash, marcos de trabajo (Django,
        Sails.js, Laravel PHP), y \LaTeX.}

  \section{Idiomas}
  \lan{Español}{Idioma Nativo}
  \lan{Inglés}{Nivel C1}
\end{document}
